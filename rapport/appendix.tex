\appendix

\section{Codes}
\matlabcode{getY}{Donne les données du test.}
\matlabcode{val}{Cette fonction calcule différentes valeurs statistiques
des notes du test.}
\matlabcode{cmp}{Cette fonction compare les CDF des approximations avec Weibull et la Normale.}
\matlabcode{mynormfit}{Cette fonction approxime $\mu$ et $\sigma$ avec la méthode des moments.}
\matlabcode{mywblfit}{Cette fonction approxime $\alpha^{1/m}$ et $m$ avec la méthode des moments.}

\matlabcode{wblmm}{Code pour la question 1.1. Cette fonction calcule un estimateur pour un échantillon d'une distribution de Weibull selon la méthode des moments.}

\matlabcode{MM_replicate}{Code pour la question 1.1. Cette routine génère un certain nombre de réplications d'échantillons, calcule des estimateurs par la méthode des moments et donne les moyennes et variances des séries d'estimateurs (et d'erreur) trouvées. Elle génère des graphes.}

\matlabcode{methode_graphique}{Code pour la question 1.2. Estimation des paramètres d'une Weibull à l'aide de la régression linéaire.}

\matlabcode{wblloglike}{Code pour la question 1.3. Cette fonction calcule la vraisemblance logarithmique à partir d'un échantillon pour une distribution de Weibull.}

\matlabcode{wblmle}{Code pour la question 1.3. Cette fonction calcule un estimateur pour un échantillon d'une distribution de Weibull selon la méthode du maximum de vraisemblance.}

\matlabcode{MLE_replicate}{Code pour la question 1.3. Cette routine génère un certain nombre de réplications d'échantillons, calcule des estimateurs par la méthode du maximum de vraisemblance et donne les moyennes et variances des séries d'estimateurs (et d'erreur) trouvées. Elle génère des graphes.}
